\documentclass[11pt,twocolumn]{article}

\usepackage[utf8]{inputenc}
\usepackage{amsmath}
\usepackage{amssymb}
\usepackage{parskip}
\usepackage{mathtools}
\usepackage{verbatim}
\usepackage{color}
\usepackage{graphicx}
\usepackage{caption}
\usepackage{listings}
\usepackage{xcolor}
\usepackage{subcaption}
\usepackage[noend,linesnumbered]{algorithm2e}
\usepackage{setspace} 
\usepackage{enumerate} 
\usepackage{amsthm}
\newcommand{\lra}{\Leftrightarrow}
\newcommand{\join}{\bowtie}
\newcommand{\select}{\sigma}
\newcommand{\project}{\pi}
\newcommand{\distinct}{\delta}
\newcommand{\Schema}{\Sigma}
\newcommand{\ra}{\rightarrow}
\newcommand{\q}{\quad}
\newcommand{\tbf}{\textbf}
\lstset { %
    language=C++,
    backgroundcolor=\color{black!5}, % set backgroundcolor
    basicstyle=\footnotesize,% basic font setting
}


\usepackage[left=1in,right=1in,top=1in,bottom=1in]{geometry}

\DeclarePairedDelimiter{\ceil}{\lceil}{\rceil}

\usepackage{amssymb}

\let\oldemptyset\emptyset
\let\emptyset\varnothing

\newcommand\NV[1]{**NV #1 ** }
\newcommand{\code}{\lstinline}



\begin{document}

\title{Project Report}
\author{Konstantinos Zarifis \quad Jules Testard \quad Michael Barrow}
\maketitle

\section{Introduction}

LLVM is a collection of modular compiler tools. The projects original goal was a flexible compilation strategy for arbitrary programming languages able to perform both static and dynamic compilation.\\
This language flexibility is mainly achieved with a common intermediate program representation during the compilation known as LLVM byte code. Source language and target machine independent optimisations can be made to the LLVM byte code, which is the 231 course project subject of interest.\\
%The price of this flexibility is complexity. End to end, using LLVM requires arranging a subset of its modules into a tool-chain sourcing a specific language and targeting a specific machine. The machine may be virtual 
Our prescribed source language is C++ and our target architecture is x86. The project goal is to profile a set of LLVM compiled C++ benchmarks programs using the \textbf{pass} feature of the \textbf{opt} LLVM module.
Three pass functionalities are required:
\begin{itemize}
\item{Collecting static instruction counts}
\item{Collecting dynamic instruction counts}
\item{Profiling branch bias }
\end{itemize}
The required depth of understanding and proficiency in LLVM module code increases incrementally with each of these three functionalities. Therefore the report is split into three mini-reports to focus on the self contained lessons learned from implementing them. Mini-reports have the following sub sections:\\
\begin{itemize}
\item{\textbf{Instrument description: }A high level description of an \textit{algorithm} used to implement functionality}
\item{\textbf{Instrument implementation summary: }Details on key LLVM concepts and api's used to implement the functionality \textit{algorithm} as an \textit{opt pass}}
\item{\textbf{Benchmark analysis: }Conjecture based running the \textit{opt pass}}
%\item{\textbf{Learning outcome: }Description of new skills acquired during the task}
\end{itemize}


\section{The interface}

We have designed the interface with the client analysis developper in mind.
It is built to be clear, understandable, easy to use and easy to extend. The focus has been put on clarity and extensibility rather than performance. Indeed, to build a basic client analysis, only two classes have to be extended and five short virtual methods overridden. We go over each component of the interface implementation. For each component, we describe how a client can use the interface and quickly build his own analysis.

\subsection{The Context Flow Graph}
The first aspect of the dataflow analysis we implemented was its internal representation, the context flow graph.
We wanted to model our dataflow analysis following closely the pseudo code found in the lecture notes. We noticed that the LLVM library provided its own graph implementation in the form of \code{BasicBlocks} and \code{Instructions}. However, we found the stucture of the LLVM library graph too complex and it did not offer the possibility to store the analysis information (which we call \emph{flow}) within. Therefore, we created our own graph structure which is defined in listing \ref{CFG}.

\begin{lstlisting}[caption=Context Flow Graph, label=CFG]
typedef struct ListNode {
  int index;
  vector<ListEdge*> incoming;
  vector<ListEdge*> outgoing;
  Instruction *inst;
  ListNode(int idx){
    index = idx;
  }
} ListNode;

typedef struct ListEdge{
  Flow* flow;
  ListNode* source;
  ListNode* destination;
  ListEdge(ListNode* src, ListNode* dst){
    source = src;
    destination = dst;
    flow = new Flow();
  }
} 
\end{lstlisting}

Each node can have multiple incoming and outgoing edges and represents an instruction in the program. An index has been added to more easily identify the instruction within the program. However, the flow is store only in the edges, in accordance with the algorithm seen in the lecture notes. A convenient method allows the user to display the graph in \code{JSON} format. Finally, the clients do not need to be aware of the graph structure, as it is only used by the class they are going to extend.

\subsection{The Flow Class}

The Flow class is the way we represent the information computed by analyses which use our interface. Any client must create his own class which will extend the flow class. The class 

\begin{itemize}
\item TOP
\item BOTTOM
\item join(Flow* other)
\item equals(Flow* other)
\item copy(Flow* other)
\item jsonString()
\end{itemize}


\subsection{The Static Analysis Class}

The dataflow analysis implementation is centered around the \tbf{worklist algorithm}, whose C++ implementation can be seen on listing \ref{worklist}.

\begin{lstlisting}[caption=Worklist Algorithm, label=worklist]
void StaticAnalysis::runWorklist() {
  queue<ListNode*> worklist;
  
  initializeEdgeFlowToBottom();
  addAllNodesToWorklist(worklist);
  
  while(!worklist.empty()){
     ListNode* current = worklist.front();
     
     Flow* in = 
     mergeFlowFromIncoming(current);
     
     Flow* out = executeFlowFunction(
     	in,current->inst);
     
     for(unsigned int i = 0 ; i < 
     	current->outgoing.size(); i++) {
       
       Flow* new_out = out->join(
       current->outgoing[i]->flow);
     
       if (!(new_out->equals(current
       ->outgoing[i]->flow))){
     
         current->outgoing[i]->f
         low->copy(new_out);
     
         worklist.push(current->
         outgoing[i]->destination);
       }
     }
    worklist.pop();
  }
}
\end{lstlisting}


We have designed the interface with the client analysis developper in mind.
It is built to be clear, understandable, easy to use and easy to extend. The focus has been put on clarity and extensibility rather than performance. Indeed, to build a basic client analysis, only two classes have to be extended and five short virtual methods overridden. We go over each component of the interface implementation. For each component, we describe how a client can use the interface and quickly build his own analysis.

\subsection{The Context Flow Graph}
The first aspect of the dataflow analysis we implemented was its internal representation, the context flow graph.
We wanted to model our dataflow analysis following closely the pseudo code found in the lecture notes. We noticed that the LLVM library provided its own graph implementation in the form of \code{BasicBlocks} and \code{Instructions}. However, we found the stucture of the LLVM library graph too complex and it did not offer the possibility to store the analysis information (which we call \emph{flow}) within. Therefore, we created our own graph structure which is defined in listing \ref{CFG}.

\begin{lstlisting}[caption=Context Flow Graph, label=CFG]
typedef struct ListNode {
  int index;
  vector<ListEdge*> incoming;
  vector<ListEdge*> outgoing;
  Instruction *inst;
  ListNode(int idx){
    index = idx;
  }
} ListNode;

typedef struct ListEdge{
  Flow* flow;
  ListNode* source;
  ListNode* destination;
  ListEdge(ListNode* src, ListNode* dst){
    source = src;
    destination = dst;
    flow = new Flow();
  }
} 
\end{lstlisting}

Each node can have multiple incoming and outgoing edges and represents an instruction in the program. An index has been added to more easily identify the instruction within the program. However, the flow is store only in the edges, in accordance with the algorithm seen in the lecture notes. A convenient method allows the user to display the graph in \code{JSON} format. Finally, the clients do not need to be aware of the graph structure, as it is only used by the class they are going to extend.

\subsection{The Flow Class}

The \code{Flow} class is the way we represent the information computed by analyses which use our interface. Any client must create his own class which will extend the \code{Flow} class. The class does not contain any container for the computed, leaving this job entirely to the subclass. This decision allows the class to be very flexible and allow just about any kind of data model for the computed information, although this information is most often represented a map from variables to analysis-specific domain elements. The subclass is however required to override a number of virtual C++ functions.

The \code{Flow} class includes a lot of the functionality from the lattice operators seen in class. The first method overriden by the client is the \code{Flow* join(Flow* other);} method, which joins the \code{this} flow and the \code{other} flow together. The \code{this} flow and the \code{other} flow objects are not altered by this method, rather a copy of their merged information is outputted. The client must also override the \code{bool equals(Flow* other)} and \code{void copy(Flow* other)} which allow comparison and duplication, respectively. The \code{void copy(Flow* other)} method should be read as "copy flow \code{other} into \code{this}", and therefore does not have a return type.

In all of the overridden versions of these methods, the client will have to cast the input \code{Flow*} type to an instance of his client class in order to have access to all the fields required to implement these methods accurately. The type casting can happen safely, because it will always be the case when those methods are called, that the incoming flow's actual type will be that of the client subclass. For increased readibility, we initially wanted the $copy$ and $equals$ methods to be implemented through operator overloading. However, we realized that this is not possible, given operator overloading restricts the input to be a constant pointer and disables casting.

The last method the client must override is the \code{string jsonString()} method which allows the output of the flow in \code{JSON format}. The class defines two constants strings, called \code{"top"} and \code{"bottom"} and a member called \code{basic}, which is only allowed to be empty or have these two values. The \code{jsonString()} method will output \code{"top"} or \code{"bottom"} if and only if the basic string is non-empty, regardless of the analysis, which help clarify the flow representation when outputed in \code{JSON} format.

\subsection{The Static Analysis Class}

While the \code{Flow} class represents the information computed by the dataflow analysis, the \code{Static Analysis} implements the analysis itself. It must also be extended by clients. The \code{Static Analysis} class implements most of the effort required for an analysis, leaving the client class focusing on the flow function implementation.

The \code{StaticAnalysis} class has the responsibility of building the Context Flow Graph, which it does in the \code{void builCFG(llvm::Function &F)} method. Recall that our analysis are made on a function scope. This method will build the graph corresponding to the LLVM function given as argument. When analysis multiple procedures, multiple instances of the Analysis class are created. More details about this are discussed in the next section.

The most important feature of the \code{StaticAnalysis} class is the \code{void runWorklist()} method, whose C++ implementation can be seen on listing \ref{worklist}. This class has been made on purpose extremely similar to the worklist algorithm seen in class. Next, we go over the important methods used by this class.

The \code{void initializeEdgeFlowToBottom()} function has a dual functionality. First, it makes sure the edge flow is initialized to bottom at the start of the analysis. Second, it ensures that the C++ type of the flow used by algorithm is that of the client analysis subclass. When the client extends this class, in his constructor he must initialize the two \code{Flow*} objects \code{top} and \code{bottom} to instance of the client analysis's \code{Flow*} subclass. Next, the \code{Flow* initialize()} method must to be overridden. This method returns a copy of the \code{bottom} flow. When the  \code{void initializeEdgeFlowToBottom()}, it will call \code{Flow* initialize()} on all edge flows in the graph. The method \code{void addAllNodesToWorklist()} is pretty self explanatory.

The \code{Flow* mergeFlowFromIncoming()} method gets all of the flows from the incoming edges of the current node and joins them together using the \code{Flow* join()} method. Note that given all \code{Flow*} objects are instance of the client subclass, through polymorphism the \code{Flow* join()} method that will be called will be that of the client subclass.

Next, the \code{Flow* executeFlowFunction()} method outputs the \code{Flow*} corresponding to the input \code{Flow*} and \code{Instruction*}. This method must be overridden by the client. Inside this function, this client is expected to implement the functionality for every flow function. This leaves the client very free in the way he intends to implement flow functions.

The last part of the worklist algorithm updates the flows of the output edges. Notice that given two consecutive nodes, the outgoing edge for the first node and incoming edge of the second node are the same object (by updating one, you update the other). This section relies on the client interfaces implementing the \code{Flow* join()}, \code{bool equals()}, \code{void copy()} methods correctly.

\begin{lstlisting}[caption=Worklist Algorithm, label=worklist]
void StaticAnalysis::runWorklist() {
  queue<ListNode*> worklist;
  
  initializeEdgeFlowToBottom();
  addAllNodesToWorklist(worklist);
  
  while(!worklist.empty()){
     ListNode* current = worklist.front();
     
     Flow* in = 
     mergeFlowFromIncoming(current);
     
     Flow* out = executeFlowFunction(
     	in,current->inst);
     
     for(unsigned int i = 0 ; i < 
     	current->outgoing.size(); i++) {
       
       Flow* new_out = out->join(
       current->outgoing[i]->flow);
     
       if (!(new_out->equals(current
       ->outgoing[i]->flow))){
     
         current->outgoing[i]->f
         low->copy(new_out);
     
         worklist.push(current->
         outgoing[i]->destination);
       }
     }
    worklist.pop();
  }
}
\end{lstlisting}

\subsection{The AnalysisPass}

Until now, it has been assumed that the LLVM code we want to analyze was given. We actually require the client to write an \code{LLVM::Pass} to perform the analysis. The user is then required to use the LLVM \code{opt} program with the \code{-analyze} flag on the pass he creates to output the computed information. Given our analyses all use function scope, the \code{Pass} should create one analysis object per function in the LLVM module. Notice that the client is not required to extend any class, but an example \code{Pass} is provided as a guideline. Finally, noticed that in our design, it is possible for a single pass to handle multiple analyses, if required. We believe extending our analysis to module scope would be easy, given the flexibility of our framework. In order to extend our scope, we only need to extend our Context Flow Graph to add edges from caller to callee functions (in addition to branching and jumps).

\section{Constant Propagation}

This analysis is again performed in an intra-procedural manner. The theoretical background for this analysis has been covered in class so we will not analyze it further in this report. Instead, we will describe how we managed to implement the flow-functions, the lattices and the actual flow representation. 


\subsection{Lattice, Flow \& Flow Functions}
As previously described, we once again utilize stl maps to represent the in and out flow of each statement. However, since this is a constant propagation analysis the domain is different. Specifically D = Powerset($\{x\ra c|x\in vars \wedge c\in constants\}$), so a map that uses a string as a key was used to keep the name of the register and a float was used as a value to keep the actual constants that this register points to. We didn't have to generate a representation of the lattice since it's implicitly described by our flow functions and the implemented methods in the Flow class.

Specifically, inside the Constant Propagation Flow class, we had to implement the equals class, which given the two flows we are trying to compare it returns true if the two contained maps hold the same elements, or false otherwise. Method jsonString outputs the flow at the end of the analysis in an easily readable json format



\section{Available Expressions}

Common subexpression elimination (CSE), is an optimization that searches for instances of identical expressions (i.e., they all evaluate to the same value), and analyses whether it is worthwhile replacing them with a single variable holding the computed value. We analyzed CSE thoroughly in class, but in order to actually perform CSE we need to have the set of available expressions for a specific program, and this is what this pass does, it provides the set of expressions that don't have to be recomputed. This analysis is also top-down as every other analysis described here.


\subsection*{Lattice, Flow \& Flow Functions}

Once again, maps were used to represent the flow informaton. Specifically in this case, 

\section{Range Analysis}

%
%\begin{table*}[t]
%\centering
%\caption{Pointer Analysis Instruction coverage}
%\begin{tabular}{| c | c | c | }
%\hline
%Instruction & C++ code & LLVM bit code  \\
%\hline
%$X \la NULL$ &
%\text{\code{Type* x = 0;}} &
%\text{\code{store Type* null, Type** \%X, align 4}} \\
%\hline
%$X \la Y + c$ & \shortstack{\code{Type* x,y;} \\ \code{int i;} \\ \code{x = y+i;}} & 
%\shortstack{ \code{\%add.ptr = getelementptr inbounds Type** \%Y, i32 i} \\
%\code{store Type** \%add.ptr, Type** \%X, align 4}}  \\
%\hline
%$X \la \&Y$ & 
%\shortstack{\code{Type y;} \\ \code{Type* x = \&y;}} &
%\text{\code{store Type* \%Y, Type** \%X, align 4}} \\
%\hline
%$*X \la Y$ &
%\shortstack{\code{Type** x;} \\ \code{Type* y;} \\ \code{*x = y;}} &
%\shortstack{\code{\%0 = load Type** \%Y, align 4} \\ 
%\code{\%1 = load Type*** \%X, align 4} \\
%\code{store Type* \%0, Type** \%1, align 4} } \\
%\hline
%$X \la *Y$ &
%\shortstack{\code{Type* x;} \\ \code{Type** y;} \\ \code{x = *y;}} &
%\shortstack{\code{\%0 = load Type*** \%Y, align 4} \\ 
%\code{\%1 = load Type** \%0, align 4} \\
%\code{store Type* \%1, Type** \%X, align 4} } \\
%\hline
%\end{tabular}
%\label{pointerAnalysisTable}
%\end{table*}

Our range analysis is intra-procedural and based on concepts introduced in class. Range analysis has similar behaviour to constant propagation in a sense, because it aims to track a variable's value through statements that modify the value. The difference being that merging of code branches will result in the a tuple representing the range of any variable appearing in both branches to be merged as shown if Figure~\ref{RngMergeNode}.\\
\begin{figure}[here]
\includegraphics[width=0.4\textwidth]{MergeNode}
\caption{A Range analysis merge. Unlike constant propagation, $x$ becomes all possible values at $in_c$ and not top. }
\label{RngMergeNode}
\end{figure}
There are two complications with this analysis which are described as follows;\\
\begin{itemize}
\item{\textbf{Flow functions:} Flow functions must assume range tuples and calculate the maximum tuple for each statement.}
\item{\textbf{Loop detection:} Given that the range of numbers is infinite, for termination a heuristic loop detector should be added to the join. We will elaborate later in the document.}
\end{itemize}
\subsection{Lattice \& Flow Functions}
for range analysis, we decompose all variable types to two numerical primitives of \emph{int} and \emph{float} that correspond to LLVM's constant and register types \emph{IntxTy} and \emph{FloatTy}. This bounds range to the \emph{max} and \emph{min} of these types, but it is still large enough that we need the heuristic loop detector mentioned earlier. Any constant or operand that cannot be cast to a \emph{int} or \emph{float} cannot have a range and so will be outside of the lattice domain.\\
We introduce the concept of:
$$ X \ra \{min,max\} \in Z^2$$
to represent the range of a variable $X$ given the set of variables $Z$, where $min$ is the lowest value X could be and $max$ is the highest. \\
Furthermore we define all constants $N$ to be a range of 1 using similar notation: 
$$ N \ra\{n,n\} \in \mathbb{R}$$
This means that constants may be assigned to variables, as is necessary for analysis and in actual code. \\
\subsubsection{Lattice and the Join}
The set join is complicated for range analysis so extra discussion is provided after defining the lattice.
Let $Vars$ be the set of C++ variables present in the program. Given two flows $F,F'$, we have :
\begin{align*}
F \sqsubseteq & F'  & \lra & & F \subseteq F' \\
\top & & \lra & & \{ X \ra \{-\infty; +\infty\} | \forall X \in Vars \}\\ 
\bot & & \lra & & \emptyset \\ 
F \sqcup F' & & \lra & & (F \ra h(F)) \cup (F'\ra h(F')) \\
F \sqcap F' & & \lra & & F \cap F'
\end{align*}

\textbf{The Join}\\
$h(X)$ is a loop detection heuristic. 
\begin{figure}[here]
\includegraphics[width=0.4\textwidth]{loopDetector}
\caption{The range analysis Join. A heuristic is used to quickly reach a fixed point for the loop body statements.}
\label{LoopDetectHeuristic}
\end{figure}

The heuristic is described a flow chart in Figure~\ref{LoopDetectHeuristic}
The heuristic can guarantee correctness but depending on Y it may can give very imprecise information. A higher $Y$ increasing the likelihood the loop control will reach its upper bound, which increases precision since a local fixed point occurs and statements from the loop body will not be added to the work list at this point. We were not aggressive with $Y$ because of an exponential analysis run-time cost. We reason a cost factor of $\bar{Loops} \times \bar{BodyLength}$ meaning $Y$ should not be too high for scalable analysis and choose 3 to demonstrate our heuristic concept. We considered a more sophistic range detection based on taken/untaken predecessor blocks, but unable to successfully implement this. More detail is provided in the implementation section.

\subsubsection{Flow Functions}
We list the instructions covered by our our range analysis in Table~\ref{rangeAnalysisTable}. For each instruction, we define $F_{IN}$ and $F_{OUT}$ as the input and output of the flow functions for each instruction, respectively. Table~\ref{rangeAnalysisTable} is a selection of flow functions to show the supported binary arithmetic and logical statements of our range analysis. We do not list a flow function for every operand type permutation (possible operand combinations of constants, $N$ and variables $X$). We do not need to do so since constants and variables exist in our domain ($\bot \subseteq n \subseteq \top \forall N$ and $\bot \subseteq x \subseteq \top \forall X$) so either $A$ or $B$ or both in Table~\ref{rangeAnalysisTable} may be substituted with $N$. For the same reason we also do not list different flow functions for The primitive types \emph{FloatTy} and \emph{IntxTy}. The flow functions all use a binary operator substitution, \emph{Combo(A,B,Op)} defined under Table~\ref{rangeAnalysisTable} to exhaustively find the maximum range of any binary operator. This could be optimized for adding and shifting cases, but we feel this would obfuscate the flow functions.\\
Besides the binary operators we define flow functions for pointer de-reference. Because our pass has no access to points-to information, to be safe our pass simply returns max range to either the assigned variable or the set of variables.  %All upper case literals are variables and all lower case literals are constants. We use shorthand for readability of our flow functions, which is expanded below the main table. 
%{\footnotesize
%\begin{align*} 
%& X \la A + B  & : & & F_{OUT} = F_{IN} - \{X \ra *\}\cup \{ X \ra \{min(Combo(A,B,+)); max(Combo(A,B,+)) \} | A \in Vars \wedge B \in Vars \} \\
%& X \la A - B & : & & F_{OUT} = F_{IN} - \{X \ra *\}\cup \{ X \ra \{min(Combo(A,B,-)); max(Combo(A,B,-)) \} | A \in Vars \wedge B \in Vars \} \\
%& X \la A \times B & : & & F_{OUT} = F_{IN} - \{X \ra *\}\cup \{ X \ra \{min(Combo(A,B,\times)); max(Combo(A,B,\times)) \} | A \in Vars \wedge B \in Vars \} \\
%& X \la A \div B & : & & F_{OUT} = F_{IN} - \{X \ra *\}\cup \{ X \ra \{min(Combo(A,B,\div)); max(Combo(A,B,\div)) \} | A \in Vars \wedge B \in Vars \} \\
%& X \la A MOD B & : & & F_{OUT} = F_{IN} - \{X \ra *\}\cup \{ X \ra \{min(Combo(A,B, \%)); max(Combo(A,B,\%)) \} | A \in Vars \wedge B \in Vars \} \\
%& X \la A LSHL B & : & & F_{OUT} = F_{IN} \cup \{W \ra Z | X \ra W \wedge Y \ra Z \} \\
%& X \la A LSHR B & : & & F_{OUT} = F_{IN} \cup \{X \ra Z | Y \ra W \wedge W \ra Z \} \\ 
%& X \la A ASHR B & : & & F_{OUT} = F_{IN} \cup \{X \ra Z | Y \ra W \wedge W \ra Z \} 
%\end{align*}

%where:
%}%

\begin{table*}[t]
\centering
\caption{Range analysis flow functions}
\begin{tabular}{| c | l | }
\hline
\multicolumn{1}{c}{\textbf{Statement}} & 
  \multicolumn{1}{c}{\textbf{Flow function}}\\\hline
$X \la A + B$  & \shortstack{$F_{OUT} = F_{IN} - \{X \ra *\}\cup$ \\$\{ X \ra \{min(Combo(A,B,+)); max(Combo(A,B,+)) \} | A \in Vars \wedge B \in Vars \}$} \\
$X \la A - B$  & \shortstack{$F_{OUT} = F_{IN} - \{X \ra *\}\cup$ \\$\{ X \ra \{min(Combo(A,B,-)); max(Combo(A,B,-)) \} | A \in Vars \wedge B \in Vars \}$} \\
$X \la A \times B$  & \shortstack{$F_{OUT} = F_{IN} - \{X \ra *\}\cup$ \\$\{ X \ra \{min(Combo(A,B,\times)); max(Combo(A,B,\times)) \} | A \in Vars \wedge B \in Vars \}$} \\
$X \la A \div B$  & \shortstack{$F_{OUT} = F_{IN} - \{X \ra *\}\cup$ \\$\{ X \ra \{min(Combo(A,B,\div)); max(Combo(A,B,\div)) \} | A \in Vars \wedge B \in Vars \}$} \\
$X \la A \% B$  & \shortstack{$F_{OUT} = F_{IN} - \{X \ra *\}\cup$ \\$\{ X \ra \{min(Combo(A,B,\%)); max(Combo(A,B,\%)) \} | A \in Vars \wedge B \in Vars \}$} \\
$X \la A LSHL B$  & \shortstack{$F_{OUT} = F_{IN} - \{X \ra *\}\cup$ \\$\{ X \ra \{min(Combo(A,B,LSHL)); max(Combo(A,B,LSHL)) \} | A \in Vars \wedge B \in Vars \}$} \\
$X \la A LSHR B$  & \shortstack{$F_{OUT} = F_{IN} - \{X \ra *\}\cup$ \\$\{ X \ra \{min(Combo(A,B,LSHR)); max(Combo(A,B,LSHR)) \} | A \in Vars \wedge B \in Vars \}$} \\
$X \la A ASHR B$  & \shortstack{$F_{OUT} = F_{IN} - \{X \ra *\}\cup$ \\$\{ X \ra \{min(Combo(A,B,ASHR)); max(Combo(A,B,ASHR)) \} | A \in Vars \wedge B \in Vars \}$} \\
$X \la *Y$ & $F_{OUT} = F_{IN} - \{X \ra *\}\cup \{X \ra \{-\infty,+\infty\}\}$\\
$*X \la Y$ & $F_{OUT} = F_{IN} - \{Z \ra * \forall Z\in Vars\}\cup \{Z \ra \{-\infty,+\infty\} \forall Z\in Vars\}\}$\\
\hline

\end{tabular}
\label{rangeAnalysisTable}

where: ${M[m_1:m_4]} \gets Combo(A,B,Op) \lra ((A_{[0]} Op B_{[0]}),(A_{[0]} Op B_{[1]}),(A_{[1]} Op B_{[0]}),(A_{[1]} Op B_{[1]}))$ 
\end{table*}

%Notice that scope of instructions covered could be larger and these out-of-scope instructions fall into two categories. Instructions such as $X \la c$ (where $X$ is a pointer variable) are explicitly forbidden in C++ language. Instructions such as $*X \ra *Y$ are legitimate candidates but where not implemented because of time constraints. 
\subsection{Implementation}
The range analysis pass is performed after a "\textbf{mem2reg}" pass. Without mem2reg, the pass must resolve variables from load and store instructions along with other complications.\\
Using mem2reg has a disadvantage that source code variables do not correspond to the .ll code input to our range analysis. We were unable to robustly transpose source code variables to our range analysis input, so our range analysis could not implement bounds checking.\\
Because of this limitation we were not aggressive with our join heuristics and chose a simple three statement counter to demonstrate the concept of heuristic loop detection for balancing pass execution time and information preciseness.\\
We provide an example snipped  for a flow function but focus discussion on our join operator. This is because our flow functions follow a similar template and because the join heuristic is responsible for a practical range analysis' achievable precision.\\
The multiplication flow function is as follows:
%Each mathematical instruction is identified by its corresponding C++ and LLVM code snippets, which are available on table \ref{pointerAnalysisTable}. When the \code{Flow* executeFlowFunction()} is executed, the instruction's opcode is parsed and the computation performed is selected using switch statements as follows :

\begin{lstlisting}[caption=Range analysis flow functions, label=PAFF]
DomEl getOpRng(DomEl a, 
               DomEl b, 
               unsigned opcode)
{
...//Declarations, sanity checks etc.
  switch(opcode) 
  {
  	...
    case MUL:

      Comb[0] = a.lower * b.lower;
      Comb[1] = a.lower * b.upper;
      Comb[2] = a.upper * b.lower;
      Comb[3] = a.upper * b.upper;
      resRange.lower = Comb[0];
      resRange.upper = Comb[0];
      while(i < 4)
      {
        if(Comb[i] < resRange.lower)
    	         resRange.lower = Comb[i];
        if(Comb[i] > resRange.upper)
    	         resRange.upper = Comb[i];
        i++;
      }
      break;
    ...
  }//end case
  return resRange;
}
\end{lstlisting}

The Join operation is composed of three parts:
\begin{itemize}
\item{\textbf{Heuristic loop detection $h(in)$: }Keep copies of the first mappings for each statements. Output is consumed by the join operation.}
\item{\textbf{Join sets $\cup$: }The set wise join operation}
\item{\textbf{Join elements: }A range analysis specific element wise join taking the maximum possible range of two domain elements}
\end{itemize}

Besides the set join, $\cup$, we provide code snippets below. Set join is omitted since it is similar to other analysis.\\

\jules{Jules : not sure if those listings are useful. Could help us save one page.}
 \begin{lstlisting}[caption=Heuristic loop detector $h(in)$, label=PAFF]
Flow* FF(Flow *in, Ins *inst, int S_id) 
{
 if(nodeCount.find(S_id)!=nodeCount.end())
 {
    //Y is set to 3
    if(nodeCount[S_id].VisitCtr >= 3)
    {
      ...
      //Set ranges that changed to TOP
      DifToTop(in, 
               nodeCount[S_id].nodeSet);
      //*in is modified
      return(in);
    }
    nodeCount[S_id].VisitCtr = 
              (nodeCount[S_id].VisitCtr+1);
  }
  else
  {
    ...
    nodeState thisNodeState;
	thisNodeState.VisitCtr = 1;
	thisNodeState.nodeSet = in;//'Old'
	nodeCount[NodeId] = thisNodeState;
  }

 \end{lstlisting}
 
 \begin{lstlisting}[caption=Per domain element join, label=PAFF]
DomEl JoinEls(DomEl* A, DomEl* B)
{
  DomEl maxAB;

  maxAB.lower = (A->lower <= B->lower) ? 
                       A->lower : B->lower;
  maxAB.upper = (A->upper >= B->upper) ? 
                       A->upper : B->upper;
  return maxAB;
}
 \end{lstlisting}

\section{Pointer Analysis}

\begin{table*}[t]
\centering
\caption{Pointer Analysis Instruction coverage}
\begin{tabular}{| c | c | c | }
\hline
Instruction & C++ code & LLVM bit code  \\
\hline
$X \la NULL$ &
\text{\code{Type* x = 0;}} &
\text{\code{store Type* null, Type** \%X, align 4}} \\
\hline
$X \la Y + c$ & \shortstack{\code{Type* x,y;} \\ \code{int i;} \\ \code{x = y+i;}} & 
\shortstack{ \code{\%add.ptr = getelementptr inbounds Type** \%Y, i32 i} \\
\code{store Type** \%add.ptr, Type** \%X, align 4}}  \\
\hline
$X \la \&Y$ & 
\shortstack{\code{Type y;} \\ \code{Type* x = \&y;}} &
\text{\code{store Type* \%Y, Type** \%X, align 4}} \\
\hline
$*X \la Y$ &
\shortstack{\code{Type** x;} \\ \code{Type* y;} \\ \code{*x = y;}} &
\shortstack{\code{\%0 = load Type** \%Y, align 4} \\ 
\code{\%1 = load Type*** \%X, align 4} \\
\code{store Type* \%0, Type** \%1, align 4} } \\
\hline
$X \la *Y$ &
\shortstack{\code{Type* x;} \\ \code{Type** y;} \\ \code{x = *y;}} &
\shortstack{\code{\%0 = load Type*** \%Y, align 4} \\ 
\code{\%1 = load Type** \%0, align 4} \\
\code{store Type* \%1, Type** \%X, align 4} } \\
\hline
\end{tabular}
\label{pointerAnalysisTable}
\end{table*}

The pointer analysis which we implemented is only intra-procedural, although it could be easily extended across procedures through our framework. The pointer analysis theoretical framework has already been discussed in class and on the midterm, but the implemented work is more involved given that a single instruction may span across multiple instructions in LLVM bitcode. Our pointer analysis is C++ oriented. That is, we analyze C++ programs, and the LLVM bitcode is only the medium upon the analysis is performed.
\subsection{Lattice \& Flow Functions}
In order to define the pointer analysis flow in C++, we introduce the concept of \emph{pointer} and \emph{non-pointer} variables. The information computed at each edge (a.k.a flow) is represented by a map from pointer variables to variables (pointer or non-pointer). When defining the lattice, we disregard the typing restrictions of C++. These will become relevant in the implementation. 
\subsubsection{Lattice}
Let $Vars$ be the set of C++ variables present in the program. Given two flows $F,F'$, we have :
\begin{align*}
F \sqsubseteq & F'  & \lra & & F \subseteq F' \\
\top & & \lra & & \{ X \ra Y | \forall X,Y \in Vars \}\\ 
\bot & & \lra & & \emptyset \\ 
F \sqcup F' & & \lra &  & F \cup F' \quad \text{(set union)} \\
F \sqcap F' & & \lra & & F \cap F' \quad \text{(set intersection)}
\end{align*}
\subsubsection{Flow Functions}
We present here the scope of instructions covered by our pointer analysis. For each instruction, we define $F_{IN}$ and $F_{OUT}$ as the input and output of the flow functions for each instruction, respectively. All upper case literals are variables and all lower case literals are constants.
{\footnotesize
\begin{align*} 
& X \la NULL & : & & F_{OUT} = F_{IN} - \{X \ra Z | Z \in Vars \} \\
& X \la Y + c & : & & F_{OUT} = F_{IN} - \{X \ra Z | Z \in Vars \} \\
& X \la \&Y & : & & F_{OUT} = F_{IN} \cup \{X \ra Y \} \\
& X \la Y & : & & F_{OUT} = F_{IN} \cup \{X \ra Z | Y \ra Z \} \\
& *X \la Y & : & & F_{OUT} = F_{IN} \cup \{W \ra Z | X \ra W \wedge Y \ra Z \} \\
& X \la *Y & : & & F_{OUT} = F_{IN} \cup \{X \ra Z | Y \ra W \wedge W \ra Z \} 
\end{align*}
}%

Notice that scope of instructions covered could be larger and these out-of-scope instructions fall into two categories. Instructions such as $X \la c$ (where $X$ is a pointer variable) are explicitly forbidden in C++ language. Instructions such as $*X \ra *Y$ are legitimate candidates but where not implemented because of time constraints. 
\subsection{Implementation}
Each mathematical instruction is identified by its corresponding C++ and LLVM code snippets, which are available on table \ref{pointerAnalysisTable}. When the \code{Flow* executeFlowFunction()} is executed, the instruction's opcode is parsed and the computation performed is selected using switch statements as follows :



\section{Benchmarks}

\input{benchmarks}

\section{Conclusion}

The project was completed succesfully . Learning outcomes were a working understanding of LLVM optimiser passes. We are capable of transforming and instrumenting code. We have developed an efficient method to develop, test and debug the LLVM source tree with a debug api and symbol indexed code base via the eclipse CDT. Understanding how to instrument code has provided a powerful tool for analysis of any optimisation heuristics applied by other opt passes. Instrumentation will allow us to both formulate and test heuristics based on the performance of compiled code.\\
To conclude, this project has provided us with a solid foundation for the second upcoming project for CSE 231


\end{document}
